\documentclass[letterpaper,11pt]{article}
\usepackage{graphicx}
\usepackage[margin=1in]{geometry}
\usepackage{tabularx}
\usepackage{amsmath}
\usepackage{enumerate}
\usepackage{hyperref}
\usepackage{verbatim}


\renewcommand{\vec}[1]{\mathbf{#1}} % Display vectors as boldface %

\let\oldhat\hat   % Also display hats as boldface
\renewcommand{\hat}[1]{\oldhat{\mathbf{#1}}} % Also display hats as boldface

\begin{document}

\title{Heat transfer from an ultrafast laser heated metal film to a substrate}
\author{Eric Landahl}
%\date{}

\maketitle

\section{Introduction}


\section{Classical Treatment}

\subsection{Perfect thermal contact}

These results are taken from Example 10.8 of Hahn and Ozisik, \emph{Heat Conduction} (Wiley, 2012).  The consider a two-layer composite slab with one semi-infinite layer as shown in Fig. XXX.   We will identify the film as region 1 and the bulk as region 2.   
The layers are presumed to be in perfect thermal contact with region 1 initially at a uniform temperature $T_0$ and region 2 at zero temperature.  Define a dimensionless temperature $\theta_i(x,t)$ as
\begin{equation}
\theta_i(x,t) = \dfrac{T_i(x,t)}{T_0} \hspace{0.5in} i = 1,2
\end{equation}
With this transformation, the heat transfer problem is written
\begin{subequations}
\begin{align}
\dfrac{\partial^2 \theta_1}{\partial x^2} &= \dfrac{1}{\alpha_1}\dfrac{\partial \theta_1}{\partial t} \hspace{0.5in} in \hspace{0.5in} 0<x<L, \; t>0 \\
\dfrac{\partial^2 \theta_2}{\partial x^2} &= \dfrac{1}{\alpha_2}\dfrac{\partial \theta_2}{\partial t} \hspace{0.5in} in \hspace{0.5in} x>L,\; t>0
\end{align}
\end{subequations}
subject to the boundary conditions
\begin{subequations}
\begin{align}
& \dfrac{\partial \theta_1}{\partial x} \Big| _{x=0} = 0 \\
& \theta_1(x=L,t) = \theta_2(x=L,t) \\
& k_1 \dfrac{\partial\theta_1}{\partial x} \Big| _{x=L} = k_2  \dfrac{\partial\theta_2}{\partial x} \Big| _{x=L} \\
& \theta_2(x \rightarrow \infty,t) \rightarrow 0
\end{align}
\end{subequations}
and the initial conditions
\begin{subequations}
\begin{align}
& \theta_1(x, t=0) = 1 \hspace{0.5in} in \hspace{0.5in} 0<x<L \\
& \theta_2(x, t=0) = 0 \hspace{0.5in} in \hspace{0.5in} L<x<\infty
\end{align}
\end{subequations}
where $k_i$ are the thermal conductivities and $\alpha_i$ are the thermal diffusivities of region 1 (film) and region 2 (bulk).  
Using Laplace rransforms, the solution for the temperature distribution in the two-layer medium is
\begin{subequations}
\begin{align}
& \dfrac{T_1(x,t)}{T_0}=1-\dfrac{1+\gamma}{2}\sum\limits_{n=0}^\infty \gamma^n \Bigg\{ \textnormal{erfc} \left[ \dfrac{(2n+1)L-x}{2\sqrt{\alpha_1 t}} \right] + \textnormal{erfc} \left[ \dfrac{(2n+1)L-x}{2\sqrt{\alpha_1 t}} \right] \Bigg\}  \\
& \dfrac{T_2(x,t)}{T_0}=\dfrac{1+\gamma}{2}\sum\limits_{n=0}^\infty \gamma^n \Bigg\{ \textnormal{erfc} \left[ \dfrac{(2nL+\mu(x-L)}{2\sqrt{\alpha_1 t}} \right] - \textnormal{erfc} \left[ \dfrac{(2n+2)L+\mu(x-L)}{2\sqrt{\alpha_1 t}} \right] \Bigg\} 
\end{align}
\end{subequations}
where the unitless parameters $\mu$ and $\gamma$ are defined by
\begin{subequations}
\begin{align}
& \mu = \sqrt{\dfrac{\alpha_1}{\alpha_2}} \\
& \beta = \dfrac{k_1}{k_2} \dfrac{1}{\mu} \\
& \gamma = \dfrac{\beta -1}{\beta+1} .
\end{align}
\end{subequations}

\subsection{Examples of the classical heat conduction result}

\subsubsection{Aluminum on Silicon}

\subsubsection{Aluminum on Gallium Arsenide}

\subsubsection{Aluminum on Germanium}

\subsubsection{Aluminum on Indium Antimonide}



\subsection{Diffusive Mismatch Correction}

\subsection{Acoustic Mismatch Correction}

\section{Microscopic Treatment}
Consider the same problem as in the previous section, but determine the second order energy correction.  You may leave your answer in the form of a summation.

\newpage

\section*{Appendix: Parameters}

\section*{Appendix: Code}

The most recent and up to date code should be obtained from \url{https://github.com/elandahl/dynamical-diffraction/tree/TRXD/strain_functions}.  This code listing is for reference only. 
 
\verbatiminput{../thermalFilm.m}

\end{document}
